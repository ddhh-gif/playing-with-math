\documentclass[11pt]{article}
\usepackage[utf8]{inputenc}
\usepackage[T1]{fontenc}
\usepackage{amsmath, amssymb, amsthm}
\usepackage{geometry}
\geometry{a4paper, margin=1in}
\usepackage{enumitem}
\usepackage{ctex}
\usepackage{hyperref}
\hypersetup{
	colorlinks=true,
	linkcolor=blue,
	urlcolor=blue,
}

\title{MIT 18.03 微分方程课程(2006年春季)\\第四讲课堂整理}
\author{}
\date{}

\begin{document}
	
	\maketitle
	
	\noindent 这份文档是 \textbf{MIT 18.03 微分方程课程(2006年春季)第四讲的课堂整理稿}。
	
	\section*{核心逻辑}
	我们只擅长解两种方程(分离变量法和线性方程)。面对其他复杂的“怪物”,我们的策略是通过\textbf{换元}把它们“降维”成我们会解的那两种。
	
	以下是基于文档内容的深度拆解,按照“第一性原理”逻辑进行了重构:
	
	\section{尺度变换 (Scaling):物理学家的“去量纲化”}
	这是最简单但对工程(特别是像机器人学)极重要的一步。
	
	\begin{itemize}[leftmargin=*]
		\item \textbf{直觉:} 物理方程充满了单位(秒、千克、开尔文)。这些单位不仅繁琐,还掩盖了方程的数学本质。
		\item \textbf{操作:}
		\begin{itemize}
			\item 引入新变量 $T_1 = T/M$,其中 $M$ 是外部常数温度。这样 $T_1$ 就变成了一个没有单位的纯数字(Dimensionless)。
			\item \textbf{Lumping Constants (常数打包):} 将一堆物理常数(如 $kM^3$)打包成一个新的常数 $k_1$。
		\end{itemize}
		\item \textbf{费曼视角:} 这样做之后,方程不再依赖具体的测量单位(摄氏度还是华氏度),只剩下纯粹的比例关系。方程变得更“干净”,参数更少。
	\end{itemize}
	
	\section{伯努利方程 (Bernoulli Equations):强行线性化}
	这是一类“伪装”成非线性的方程。
	
	\begin{itemize}[leftmargin=*]
		\item \textbf{形式:} $y' + p(x)y = q(x)y^n$。
		\begin{itemize}
			\item 如果 $n=0$ 或 $1$,它是线性的或可分离的,很简单。
			\item 如果 $n$ 是其他数(比如 $2, -5, 1/2$),原本的线性结构被 $y^n$ 破坏了。
		\end{itemize}
		\item \textbf{破解策略(Hack):}
		\begin{enumerate}
			\item \textbf{除法攻击:} 不要盯着 $y^n$ 发愁,直接全式除以 $y^n$。方程变为:
			\[
			\frac{y'}{y^n} + p(x)\frac{1}{y^{n-1}} = q(x)
			\]
			\item \textbf{发现规律:} 你会发现 $\frac{1}{y^{n-1}}$ 的导数恰好和第一项 $\frac{y'}{y^n}$ 有关(差一个常数倍数)。
			\item \textbf{换元:} 令 $v = y^{1-n}$(或者 $1/y^{n-1}$)。
		\end{enumerate}
		\item \textbf{结果:} 方程神奇地变成了一个关于 $v$ 的\textbf{线性方程}。
		\item \textbf{讲师建议:} 别背公式,记住这个“除法”的操作逻辑。
	\end{itemize}
	
	\section{齐次方程 (Homogeneous Equations):缩放对称性}
	这里的“齐次”不是指 $Ax=0$,而是指一种几何上的\textbf{对称性}。
	
	\begin{itemize}[leftmargin=*]
		\item \textbf{几何直觉 (Zoom Invariance):} 如果把 $x$ 和 $y$ 轴同时放大 $a$ 倍(Zoom),方程的形式保持不变。这意味着斜率 $y'$ 只取决于 $y$ 和 $x$ 的比值,即 $y' = F(y/x)$。
		\item \textbf{陷阱与技巧:}
		\begin{itemize}
			\item 直觉上你会想令 $z = y/x$。但是如果你直接算 $z'$,会非常麻烦。
			\item \textbf{逆向换元 (Inverse Substitution):} 使用 $y = zx$。
			\item 因为我们更擅长处理乘积的导数:$y' = z'x + z$。
		\end{itemize}
		\item \textbf{结果:} 方程转化为可分离变量方程:
		\[
		x \frac{dz}{dx} = F(z) - z.
		\]
	\end{itemize}
	
	\section{极坐标:毒贩船与灯塔 (The Drug Boat)}
	这是一个非常经典的“追踪曲线”问题,很有机器人路径规划的味道。
	
	\begin{itemize}[leftmargin=*]
		\item \textbf{场景:} 一艘毒贩船被灯塔照住。为了逃跑,船始终保持与光束成 $45^\circ$ 角航行。
		\item \textbf{建模:}
		\begin{itemize}
			\item 我们需要求船的轨迹 $y(x)$。
			\item 船的航向(斜率 $y'$)是光束角度 $\alpha$ 加上 $45^\circ$。
			\item 利用三角函数公式:$y' = \tan(\alpha + 45^\circ) = (\tan\alpha + 1)/(1 - \tan\alpha)$。
			\item 因为 $\tan\alpha = y/x$,所以方程是:$y' = \frac{y/x + 1}{1 - y/x}$。
		\end{itemize}
		\item \textbf{求解与顿悟:}
		\begin{itemize}
			\item 这是一个典型的齐次方程(右边全是 $y/x$)。
			\item 通过繁琐的积分,最后得到隐函数解。
			\item \textbf{结尾:} 讲师最后指出,如果在直角坐标系下看这个解简直是“一团糟” (a mess)。但如果转到\textbf{极坐标} (Polar Coordinates),利用 $\tan^{-1}(y/x) = \theta$,解竟然是惊人简洁的\textbf{对数螺线 (Logarithmic Spiral)}:
			\[
			r = c_1 e^\theta.
			\]
		\end{itemize}
	\end{itemize}
	
	\section*{总结}
	这篇文档实际上是在教你如何通过\textbf{变换视角}来解题:
	\begin{enumerate}[label=\arabic*.]
		\item \textbf{Scaling:} 消除单位的干扰,看清物理本质。
		\item \textbf{Bernoulli:} 通过除法暴露隐含的线性结构。
		\item \textbf{Homogeneous:} 利用几何上的缩放对称性。
		\item \textbf{Polar Coordinates:} 对于旋转对称的问题(如灯塔),选对坐标系(极坐标)比死算微积分更重要。
	\end{enumerate}
	
	\section{4 个核心问题的完整题面}
	
	\subsection{问题 1:高温冷却模型的去量纲化 (Scaling)}
	\textbf{背景:} 描述物体在极高温度下的冷却过程(类似 Stefan-Boltzmann 定律),此时牛顿冷却定律不再适用。
	
	\textbf{题面:}
	已知高温冷却方程如下:
	\[
	\frac{dT}{dt} = k(M^4 - T^4)
	\]
	其中:
	\begin{itemize}
		\item $T$ 是物体内部温度(变量)。
		\item $M$ 是外部环境恒定温度(常数)。
		\item $k$ 是物理常数。
	\end{itemize}
	
	\textbf{任务:} 通过引入新变量 $T_1 = T/M$ 对该方程进行尺度变换(Scaling),使其去量纲化并简化常数。
	
	\subsection{问题 2:伯努利方程求解 (Bernoulli Equation)}
	\textbf{背景:} 作为一个具体的数学练习,展示如何通过换元法解非线性方程。
	
	\textbf{题面:}
	求解以下一阶微分方程:
	\[
	y' = \frac{y}{x} - y^2
	\]
	(注:该方程也可以写成 $xy' - y = -xy^2$ 的形式)
	
	\textbf{任务:} 识别其为伯努利方程(Bernoulli Equation),并求出通解。
	
	\subsection{问题 3:毒贩船的追踪轨迹 (The Drug Boat / Pursuit Curve)}
	\textbf{背景:} 这是一个经典的追踪曲线问题。一艘船试图逃离灯塔的光束。
	
	\textbf{题面:}
	假设灯塔位于原点 $(0,0)$。一艘毒贩船被灯塔的光束照中。为了逃跑,船的航行方向(切线方向)始终与光束(径向方向)保持 $45^\circ$ 的夹角。
	
	\textbf{任务:}
	\begin{enumerate}
		\item 建立描述船的轨迹 $y(x)$ 的微分方程(提示:利用 $\tan(\alpha + 45^\circ)$ 公式)。推导出的方程应为:
		\[
		y' = \frac{x+y}{x-y}.
		\]
		\item 求解该轨迹方程(推荐使用极坐标变换)。
	\end{enumerate}
	
	\subsection{问题 4:非线性旋转动力系统分析 (Phase Portrait)}
	\textbf{背景:} 考察二维动力系统的相图绘制和流的性质。
	
	\textbf{题面:}
	已知以下常微分方程组(ODE):
	\[
	\begin{aligned}
		\dot{x} &= -y(x^2 + y^2) \\
		\dot{y} &= x(x^2 + y^2)
	\end{aligned}
	\]
	
	\textbf{任务:}
	\begin{enumerate}
		\item \textbf{Sketch the phase portrait:} 手绘相图。
		\item \textbf{Find the flow $\phi_t$:} 求出该系统的流(即解的解析式)。
		\item \textbf{Find the orbit $O(x_0)$:} 找出过初始点 $x_0$ 的轨道。
		\item \textbf{Find the limit set $\omega(x_0)$:} 找出该轨道的极限集。
	\end{enumerate}
	
	\section{从圆方程到微分方程}
	
	\subsection{为何 $x^2+y^2=R^2$ 在微分方程里不好使}
	简单直接的回答是:\textbf{你是对的。光看圆的方程 $x^2 + y^2 = R^2$,你确实推不出 $\dot{x}=-y, \dot{y}=x$ 这个特定的“速度”。}
	
	圆方程只能告诉你\textbf{“路在哪里”},但没告诉你\textbf{“车开多快”},也没告诉你\textbf{“车往哪头开”}。
	
	我们可以从以下三个层次把这个逻辑彻底打通:
	
	\subsubsection*{1. 静态 vs 动态(轨道 vs 速度)}
	\begin{itemize}
		\item \textbf{圆方程 ($x^2 + y^2 = R^2$):} 这是一张\textbf{地图}。它告诉你:“嘿,所有的点都必须在这个圈圈上。”这叫\textbf{几何约束}。它没有时间 $t$ 的概念。
		\item \textbf{微分方程 ($\dot{x}=-y, \dot{y}=x$):} 这是\textbf{驾驶指南}。它告诉你:“现在的水平速度是负的纵坐标,垂直速度是正的横坐标。”
	\end{itemize}
	
	\textbf{关键点:} 同一个圆方程 $x^2 + y^2 = R^2$,可以对应无数种微分方程!
	\begin{itemize}
		\item $\dot{x} = -y, \ \dot{y} = x$ (你是这个,逆时针,角速度1)
		\item $\dot{x} = -2y, \ \dot{y} = 2x$ (也是圆,但是转得快两倍)
		\item $\dot{x} = y, \ \ \dot{y} = -x$ (也是圆,但是顺时针转)
		\item $\dot{x} = -y(x^2+y^2), \ \dot{y} = x(x^2+y^2)$ (就是你题目的那个,也是圆,但是离圆心越远转得越快)
	\end{itemize}
	所以,\textbf{圆方程是“果”,微分方程是“因”。} 你不能从唯一的“果”反推唯一的“因”。
	
	\subsubsection*{2. 它们在哪里“会师”?(切线斜率)}
	虽然圆方程推不出具体的速度大小,但它能限制速度的\textbf{方向}。我们来验证一下它们是否“吻合”。
	
	\textbf{方法 A:从圆方程出发(隐函数求导)}\\
	对 $x^2 + y^2 = R^2$ 两边求微分:
	\[
	2x dx + 2y dy = 0
	\]
	\[
	\frac{dy}{dx} = -\frac{x}{y}.
	\]
	这是几何上\textbf{切线的斜率}。
	
	\textbf{方法 B:从微分方程出发(速度比值)}\\
	你有 $\dot{x} = -y$ 和 $\dot{y} = x$。\\
	根据链式法则,轨迹的斜率是纵向速度除以横向速度:
	\[
	\frac{dy}{dx} = \frac{dy/dt}{dx/dt} = \frac{x}{-y} = -\frac{x}{y}.
	\]
	
	\textbf{结论:} 你看!\textbf{$\frac{dy}{dx}$ 是一模一样的!} 这说明微分方程 $\dot{x}=-y, \dot{y}=x$ 确实是\textbf{贴着}圆的切线在跑的。圆方程虽然没给出速度 $\dot{x}$ 和 $\dot{y}$ 的具体表达式,但它严格规定了 $\dot{y}/\dot{x}$ 必须等于 $-x/y$。
	
	\subsubsection*{3. 如何从圆方程“凑”出这个微分方程?(参数化)}
	如果你非要从圆方程推导出这个微分方程,你必须引入\textbf{时间参数 $t$},并做一个\textbf{参数化 (Parameterization)}的假设。
	
	\begin{enumerate}
		\item \textbf{看着圆方程:} $x^2 + y^2 = R^2$。
		\item \textbf{引入参数:} 我们知道三角函数 $\cos^2 t + \sin^2 t = 1$ 是圆的完美描述。所以我们假设:
		\[
		x(t) = R \cos(t), \quad y(t) = R \sin(t)
		\]
		(注意:这里我默认了角速度 $\omega=1$,这是我为了凑出你的方程特意选的)
		\item \textbf{对时间求导:}
		\[
		\dot{x} = \frac{d}{dt}(R \cos t) = -R \sin t, \quad \dot{y} = \frac{d}{dt}(R \sin t) = R \cos t.
		\]
		\item \textbf{代回变量:} 你看 $\dot{x}$ 的结果里,$-R \sin t$ 不就是 $-y$ 吗?你看 $\dot{y}$ 的结果里,$R \cos t$ 不就是 $x$ 吗?
		\[
		\implies \dot{x} = -y, \quad \dot{y} = x.
		\]
	\end{enumerate}
	
	\textbf{真相大白:} 这一组微分方程,本质上就是\textbf{“参数方程 $x=R\cos t, y=R\sin t$ 的微分形式”}。
	
	\subsubsection*{总结}
	你觉得“搞不出这个玩意”是因为:
	\begin{itemize}
		\item \textbf{圆方程}只描述了\textbf{形状}(空间信息)。
		\item \textbf{微分方程}描述了\textbf{形状 + 速度}(空间 + 时间信息)。
	\end{itemize}
	要把圆方程变成微分方程,你必须\textbf{人为地}赋予它一个运动规律(比如:假设它是匀速转动的)。
	
	\subsection{矩阵描述圆方程}
	我们用三种视角来“拆解”它:\textbf{几何直觉}、\textbf{极坐标推导}和\textbf{线性代数视角}。
	
	\subsubsection*{1. 物理/几何直觉:速度与位置垂直}
	这是最“费曼”的解释方法。想象你在平面上驾驶一个小车。
	
	\begin{itemize}[leftmargin=*]
		\item \textbf{位置向量:} $\vec{r} = \begin{bmatrix} x \\ y \end{bmatrix}$
		\item \textbf{速度向量:} $\vec{v} = \begin{bmatrix} \dot{x} \\ \dot{y} \end{bmatrix} = \begin{bmatrix} -y \\ x \end{bmatrix}$
	\end{itemize}
	
	我们要问两个问题:
	1.  \textbf{我在远离圆心吗?}(看速度是否和半径垂直)
	2.  \textbf{我在往哪边转?}(看具体的方向)
	
	\subsubsection*{2. 数学推导:极坐标验证}
	证明它是“匀速”且“圆周”。
	
	令 $x = r \cos \theta, y = r \sin \theta$。
	
	\paragraph{证明 $r$ 不变 (圆周)}
	$$r^2 = x^2 + y^2$$
	两边对时间 $t$ 求导:
	$$2r\dot{r} = 2x\dot{x} + 2y\dot{y}$$
	代入 $\dot{x}=-y, \dot{y}=x$:
	$$r\dot{r} = x(-y) + y(x) = 0$$
	由于 $r \neq 0$,所以 $\dot{r} = 0$。\textbf{半径恒定。}
	
	\paragraph{证明 $\dot{\theta}$ 恒定 (匀速)}
	利用角度公式 $\tan \theta = \frac{y}{x}$,或者直接用角速度公式:
	$$\dot{\theta} = \frac{x\dot{y} - y\dot{x}}{x^2 + y^2}$$
	代入方程:
	$$\dot{\theta} = \frac{x(x) - y(-y)}{r^2} = \frac{x^2 + y^2}{r^2} = \frac{r^2}{r^2} = 1$$
	\textbf{结论:}
	\begin{itemize}
		\item $\dot{\theta} = 1$:角速度是常数 1 (rad/s),所以是\textbf{匀速}。
		\item 符号为正:按照数学约定,正角速度代表\textbf{逆时针}。
	\end{itemize}
	
	\subsubsection*{3. 线性代数视角:旋转生成元 (Robotics 视角)}
	写成矩阵形式:
	$$\begin{bmatrix} \dot{x} \\ \dot{y} \end{bmatrix} = \underbrace{\begin{bmatrix} 0 & -1 \\ 1 & 0 \end{bmatrix}}_{A} \begin{bmatrix} x \\ y \end{bmatrix}$$
	
	\paragraph{A. 特征值分析}
	求矩阵 $A$ 的特征值 $\lambda$:
	$$\det(A - \lambda I) = \lambda^2 + 1 = 0 \implies \lambda = \pm i$$
	\textbf{纯虚数特征值}对应于\textbf{纯振荡(旋转)}。
	\begin{itemize}
		\item 实部为 0 $\to$ 没有衰减(不向内旋),没有增长(不向外旋)。
		\item 虚部为 1 $\to$ 频率为 1。
	\end{itemize}
	
	\paragraph{B. 矩阵指数 (Matrix Exponential)}
	这个微分方程的解是:
	$$\mathbf{x}(t) = e^{At} \mathbf{x}(0)$$
	回顾泰勒展开或欧拉公式,这个反对称矩阵(Skew-symmetric matrix)的指数就是\textbf{旋转矩阵}:
	$$e^{\begin{bmatrix} 0 & -1 \\ 1 & 0 \end{bmatrix} t} = \begin{bmatrix} \cos t & -\sin t \\ \sin t & \cos t \end{bmatrix}$$
	这正是将初始向量 $\mathbf{x}(0)$ \textbf{逆时针旋转 $t$ 弧度}的操作。
	
	\subsubsection*{总结}
	\begin{enumerate}
		\item \textbf{点积为0}:速度垂直于半径 $\to$ 圆周运动。
		\item \textbf{角速度为正}:$\dot{\theta}=1$ $\to$ 逆时针。
		\item \textbf{矩阵视角}:矩阵 $\begin{bmatrix} 0 & -1 \\ 1 & 0 \end{bmatrix}$ 是二维旋转群 $SO(2)$ 的\textbf{李代数生成元}。
	\end{enumerate}
	
	\section{问题解答}
	
	\subsection{问题1:高温冷却模型的去量纲化 (Scaling)}
	% 解答暂略
	
	\subsection{问题2:伯努利方程求解 (Bernoulli Equation)}
	% 解答暂略
	
	\subsection{问题3:毒贩船的追踪轨迹 (The Drug Boat / Pursuit Curve)}
	这正是文档结尾讲师提到的那个“极坐标变换”的威力所在。用直角坐标(Cartesian Coordinates)解这个问题是一场噩梦(文档里用了两页纸),但用极坐标(Polar Coordinates)只需要三行。
	
	这是典型的\textbf{费曼式简化}:选择正确的坐标系,问题就解决了一半。
	
	\subsubsection*{1. 物理直觉图景}
	首先,我们画出极坐标下的速度分解图。
	在极坐标系 $(r, \theta)$ 中,任何运动的微元位移 $d\vec{s}$ 都可以分解为两个垂直的方向:
	\begin{enumerate}
		\item \textbf{径向位移 (Radial):} $dr$(沿着光束方向跑的距离)。
		\item \textbf{切向位移 (Tangential):} $r d\theta$(垂直于光束方向跑的弧长)。
	\end{enumerate}
	
	\textbf{核心约束条件:}
	题目中说“船始终与光束保持 $45^\circ$ 角”。
	这意味着,速度矢量 $\vec{v}$ 与径向矢量 $\vec{r}$ 的夹角 $\alpha = 45^\circ$。
	
	\subsubsection*{2. 推导过程}
	\paragraph{第一步:列出几何关系}
	在微小的三角形中,正切值 $\tan(\alpha)$ 等于“对边”(切向位移)除以“邻边”(径向位移):
	$$\tan(\alpha) = \frac{\text{切向位移}}{\text{径向位移}} = \frac{r d\theta}{dr}$$
	
	\paragraph{第二步:代入物理条件}
	已知 $\alpha = 45^\circ$,且 $\tan(45^\circ) = 1$。
	所以方程瞬间简化为:
	$$\frac{r d\theta}{dr} = 1$$
	
	\paragraph{第三步:分离变量与积分}
	这比直角坐标那个复杂的 $y' = \frac{x+y}{x-y}$ 简单太多了。我们要解的是:
	$$d\theta = \frac{dr}{r}$$
	
	两边积分:
	$$\int d\theta = \int \frac{dr}{r}$$
	$$\theta = \ln(r) + C$$
	或者写成讲师提到的形式:
	$$\ln(r) = \theta - C$$
	
	\paragraph{第四步:整理结果}
	对两边取指数(Exponentiate):
	$$r = e^{\theta - C} = e^{-C} \cdot e^\theta$$
	令常数 $c_1 = e^{-C}$,我们要找的轨迹就是:
	$$r = c_1 e^\theta$$
	
	\subsubsection*{3. 结论:对数螺线 (Logarithmic Spiral)}
	这就是著名的\textbf{对数螺线}(等角螺线)。
	\begin{itemize}
		\item \textbf{直觉含义:} 随着角度 $\theta$ 线性增加(转圈),半径 $r$ 指数级爆炸增长。这就是为什么毒贩船能逃脱的原因——它离原点越远,横向逃跑的速度($r d\theta$)就越快,逃离的效率呈指数级上升。
		\item \textbf{自然界的对应:} 这种形状在自然界中随处可见,比如\textbf{鹦鹉螺的壳}、\textbf{台风的云系}、甚至\textbf{银河系的旋臂}。它们的共同点和这艘船一样:\textbf{生长/扩张的方向始终与径向保持固定角度。}
	\end{itemize}
	
	文档的最后讲师感叹:“如果一开始就用极坐标,没人会觉得这是个难题。”  这就是坐标系选择的艺术。
	
	\subsection{问题4:非线性旋转动力系统分析 (Phase Portrait)}
	\subsubsection*{1. 物理直觉 (Feynman's Intuition)}
	观察方程组:
	\[
	\begin{aligned}
		\dot{x} &= -y(x^2 + y^2) \\
		\dot{y} &= x(x^2 + y^2)
	\end{aligned}
	\]
	
	\begin{itemize}[leftmargin=*]
		\item \textbf{看结构:} 如果把括号里的 $(x^2+y^2)$ 遮住,剩下 $\dot{x}=-y, \dot{y}=x$。这是一个标准的\textbf{匀速圆周运动}(逆时针旋转)。
		\item \textbf{看系数:} 系数 $(x^2+y^2)$ 正好是半径的平方 $r^2$。
		\item \textbf{结论:} 这是一个旋转系统。粒子在做圆周运动,\textbf{半径 $r$ 不变},但是旋转的\textbf{角速度 $\omega$} 不是常数,它等于 $r^2$。也就是说,\textbf{离原点越远,转得越快}。
	\end{itemize}
	
	\subsubsection*{2. 严谨推导:极坐标变换}
	为了证明上述直觉,并求出具体的流(Flow),我们使用极坐标换元:
	$$x = r \cos \theta, \quad y = r \sin \theta$$
	$$x^2 + y^2 = r^2$$
	
	我们需要求 $\dot{r}$ 和 $\dot{\theta}$。
	
	\paragraph{第一步:径向变化 $\dot{r}$}
	$$r^2 = x^2 + y^2$$
	两边对时间 $t$ 求导:
	$$2r\dot{r} = 2x\dot{x} + 2y\dot{y}$$
	代入原方程:
	$$r\dot{r} = x[-y(r^2)] + y[x(r^2)] = -xyr^2 + xyr^2 = 0$$
	$$r\dot{r} = 0 \implies \dot{r} = 0$$
	
	\textbf{物理含义:} 只要 $r \neq 0$,径向速度为 0。粒子永远在同一个圆上运动,不会向内收缩也不会向外扩散。
	
	\paragraph{第二步:角度变化 $\dot{\theta}$}
	利用公式 $\theta = \arctan(y/x)$ 或 $x\dot{y} - y\dot{x} = r^2\dot{\theta}$ (角动量形式)。我们用后者:
	$$\dot{\theta} = \frac{x\dot{y} - y\dot{x}}{x^2 + y^2}$$
	代入原方程:
	$$\dot{\theta} = \frac{x[x(r^2)] - y[-y(r^2)]}{r^2} = \frac{r^2(x^2 + y^2)}{r^2} = \frac{r^2 \cdot r^2}{r^2} = r^2$$
	
	\textbf{最终的简化方程组:}
	\[
	\begin{cases}
		\dot{r} = 0 \\
		\dot{\theta} = r^2
	\end{cases}
	\]
	
	\subsubsection*{3. 解答四个问题}
	\paragraph{(1) Sketch the Phase Portrait (手动绘制相图)}
	\begin{itemize}
		\item \textbf{形状:} 因为 $\dot{r}=0$,轨迹是无数个\textbf{同心圆}。
		\item \textbf{原点:} $(0,0)$ 是唯一的平衡点(Equilibrium Point),因为此时 $\dot{x}=0, \dot{y}=0$。
		\item \textbf{方向:} 因为 $\dot{\theta} = r^2 > 0$,所有轨迹都是\textbf{逆时针}旋转。
		\item \textbf{速度分布:} 越靠近原点,转得越慢($\dot{\theta}$ 小);越远转得越快。
	\end{itemize}
	
	\textbf{手绘指南:}
	\begin{enumerate}
		\item 画一个中心点(原点)。
		\item 画几个同心圆。
		\item 在圆上画箭头,指向逆时针方向。
	\end{enumerate}
	
	\paragraph{(2) Find the Flow $\phi_t$ (流)}
	我们需要解出 $r(t)$ 和 $\theta(t)$。
	设初始状态为 $x_0$,对应极坐标 $(r_0, \theta_0)$。
	
	\begin{itemize}
		\item 由 $\dot{r} = 0$ 得:$r(t) = r_0 = \sqrt{x_0^2 + y_0^2}$。
		\item 由 $\dot{\theta} = r_0^2$ (因为 $r$ 是常数) 得:$\theta(t) = r_0^2 t + \theta_0$。
	\end{itemize}
	
	将其转回笛卡尔坐标:
	\[
	\begin{aligned}
		x(t) &= r_0 \cos(r_0^2 t + \theta_0) \\
		y(t) &= r_0 \sin(r_0^2 t + \theta_0)
	\end{aligned}
	\]
	这就是流 $\phi_t(x_0, y_0)$ 的解析表达式。
	如果你想写成矩阵形式(旋转矩阵),那就是:
	$$\phi_t \begin{pmatrix} x_0 \\ y_0 \end{pmatrix} = \begin{pmatrix} \cos(r_0^2 t) & -\sin(r_0^2 t) \\ \sin(r_0^2 t) & \cos(r_0^2 t) \end{pmatrix} \begin{pmatrix} x_0 \\ y_0 \end{pmatrix}$$
	其中 $r_0^2 = x_0^2 + y_0^2$。
	
	\paragraph{(3) The Orbit $O(x_0)$ (轨道)}
	轨道是指流在相空间中走过的路径集合(不考虑时间,只考虑几何形状)。
	\begin{itemize}
		\item \textbf{情况 A:} 如果 $x_0 = (0,0)$,轨道就是一个点 $O(0) = \{ (0,0) \}$。
		\item \textbf{情况 B:} 如果 $x_0 \neq 0$,轨道是以原点为圆心,半径为 $r_0 = \|x_0\|$ 的圆。
		$$O(x_0) = \{ (x, y) \in \mathbb{R}^2 \mid x^2 + y^2 = \|x_0\|^2 \}$$
	\end{itemize}
	
	\paragraph{(4) The Limit Set $\omega(x_0)$ (极限集)}
	极限集是指当 $t \to \infty$ 时,轨迹趋近的集合。
	由于这是一个周期运动(Periodic Motion),粒子会一遍又一遍地扫过整个圆。它不会收敛到一个点,也不会跑向无穷远。
	因此,\textbf{极限集就是轨道本身}。
	$$\omega(x_0) = O(x_0)$$
	即半径为 $\|x_0\|$ 的圆(对于非零初始点)。
	
	\subsubsection*{4. 这里的“陷阱”与思考}
	这道题虽然简单,但有一个很有趣的性质:\textbf{剪切 (Shear)}。
	虽然大家都在转圈,但外圈转得快,内圈转得慢。
	这就导致一个有趣的现象:如果你在 $t=0$ 时刻在 $x$ 轴正半轴画一条线段(一排粒子),过一段时间后,这条线段会卷成蚊香状(螺旋状),因为外面的粒子跑得比里面的快得多。这就是流体力学中常见的\textbf{相位混合 (Phase Mixing)} 现象的雏形。
	
\end{document}
